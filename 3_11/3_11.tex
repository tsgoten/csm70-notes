\documentclass[10pt, twocolumn]{article}
\author{Tarang Srivastava}
\usepackage{amsmath, amssymb, amsthm, commath, chngcntr, enumitem, multirow, thmtools, xcolor}
\usepackage{graphicx}
\usepackage[margin=.25in]{geometry}
\setlength{\columnsep}{.5in}
\newcommand{\C}{\mathbb{C}}
\newcommand{\question}[1]{\textcolor{blue}{#1} \\}
\newcommand{\R}{\mathbb{R}}
\newcommand{\F}{\mathbb{F}}
\newcommand{\N}{\mathbb{N}}
\newcommand{\LinearMap}[2]{\mathcal{L}(#1, #2)}
\newcommand{\poly}[2]{\mathcal{P}_{#1}\left(#2\right)}
\newcommand{\vspan}[1]{\text{span}\left(#1\right)}
\newcommand{\inv}[1]{#1^{-1}}
\newcommand{\todo}[1]{\textcolor{red}{TODO: #1} \\}
\newcommand{\nul}{\text{null }}
\newcommand{\nullity}{\text{nullity }}
\newcommand{\range}{\text{range }}
\newcommand{\rank}{\text{rank }}
\newcommand{\annhilator}[1]{#1^{0}}
\newcommand{\makechaptertitle}[1]{
\begin{center}
	\begin{large}
		#1
	\end{large}
	\begin{small}
		\\Tarang Srivastava
	\end{small}
\end{center}
}
\declaretheoremstyle[
spaceabove=\topsep, spacebelow=\topsep,
headfont=\normalfont\bfseries,
notefont=\bfseries, notebraces={Problem }{},
bodyfont=\normalfont,
postheadspace=0.5em,
name={\ignorespaces},
numbered=no,
headpunct=:]
{mystyle}
\declaretheorem[style=mystyle]{q}

\begin{document}

\makechaptertitle{CSM Section 1}

\section{Logistics}
We are going online now. We will have section in this format moving on Zoom.
The Github for this section and all the notes will be on https://github.com/tsgoten/csm70-notes.

\section{Message Corruption}
So lets say we are sending message of length $ m $. So, we will sending soemhting eventually of length $ n $. Once we send $n $ we say that 20\% will be corrupted. How long should we make $ n $.  
So the question is rephrased to being how long is $ n $.

\begin{align*} 
	n - pn = m \\
	n = \frac{m}{1 -p}
\end{align*}
For Berlekamp welch
\begin{align*} 
	n - p(2n) = m \\
	n = \frac{m}{1 - 2p}
\end{align*}

\section{Berlekamp-Welch}

\end{document}