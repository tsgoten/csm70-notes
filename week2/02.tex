
\documentclass[10pt, twocolumn]{article}
\author{Tarang Srivastava}
\usepackage{amsmath, amssymb, amsthm, chngcntr, enumerate, multirow, xcolor}
\usepackage{graphicx, hyperref}
\usepackage[margin=.25in]{geometry}
\setlength{\columnsep}{.25in}
\newcommand{\makechaptertitle}[1]{
\begin{center}
	\begin{large}
		#1
	\end{large}
	\begin{small}
		\\Tarang Srivastava
	\end{small}
\end{center}
}
\newcommand{\question}[1]{\textcolor{blue}{#1} \\}
\newcommand{\R}{\mathbb{R}}
\newcommand{\N}{\mathbb{N}}
\newcommand{\inv}[1]{#1^{-1}}
\newcommand{\todo}[1]{\textcolor{red}{TODO: #1} \\}
\theoremstyle{definition} 
\newtheorem{q}{}
\theoremstyle{definition} 
\newtheorem{lemma}{Lemma}

\begin{document}
\makechaptertitle{CSM 70 Week 2}

\section{Introduction}
    Welcome to CSM 70! I will try to do writeups like this every week. 
    It is week 2, because last week was drop-in sections and that was week 1.
    I will most likely just write down what we went over. 
    The goal is to mention everything we went over and hopefully you can find the corresponding theorems in the notes. 
\section{Graph Definitions}
Check the following definitons in the notes: \textbf{Vertices, Edges, Degree, Graph}. 
Also check the definitions for \textbf{Eulerian Tour} \\
From the worksheet $ V = \{0, 1\}^{n-1} $ is the set of all $ n -1 $ bit strings. 
\section{Non-Planar Graphs}
\href{https://youtu.be/xBkTIp6ajAg}{Video on $K_{33}$ and $ K_{5}$ }
https://youtu.be/xBkTIp6ajAg \\

\section{Trees}
What makes trees a special kind of graph? 

\section{Hypercubes}


\end{document}